\documentclass[12pt,a4paper,spanish]{article}
\usepackage{babel}
\usepackage[latin1]{inputenc}
\usepackage[pdftex]{graphicx}

\begin{document}

\begin{figure}
  \centering
    \includegraphics{C:/Users/fjgimon/Downloads/logo.jpg}
     \\ Universidad Sim\'on Bol\'ivar
     \\ Depto. De Computaci\'on y Tecnolog\'ia de la Informaci\'on
     \\ CI5311: Paradigmas de Modelado de Bases de Datos
\end{figure} 

\title{Universo del Discurso}
\author{Grace Gim\'on 08-10437 \\
        Jes\'us Mart\'inez \\
        Julio De Abreu 05-38072}

\maketitle

\newpage

\indent Se quiere implementar una Base de Datos para la p\'agina Web
\emph{USBline}, a desarrollarse por la Universidad Sim\'on Bol\'ivar
para la gesti\'on y pago en l\'inea de servicios ofrecidos a toda la
comunidad universitaria y externa.
\newline
\newline
\indent La universidad y organizaciones terceras prestan numerosos servicios, pero la comunidad atendida es extensa
 dificultando una prestaci\'on eficiente en todo momento de estos. Por lo tanto la 
incorporaci\'on de un sistema web para la realizaci\'on de una solicitud o pago de un servicio 
agilizar\'ia y facilitar\'ia el d\'ia a d\'ia de todos aquellos que hacen vida 
en la universidad. Para ello, es imprescindible la creaci\'on de una base de datos, 
ya que s\'olo a trav\'es de ella se podr\'an almacenar integralmente los
datos de todos los servicios involucrados, permitiendo adem\'as realizar consultas relevantes
a los interesados.
\newline
\newline
\indent Un servicio es un conjunto de actividades que buscan
satisfacer las necesidades de una comunidad en particular. Un servicio
puede ser pago o gratuito. Dentro de la USB existen dos tipos de
servicios: Servicios ofrecidos por la misma universidad, y servicios
ofrecidos por terceros dentro de la misma.

\section{Punto de Vista de la Universidad}

\indent Dentro de la USB existen dependencias que ofrecen distintos
servicios a la comunidad universitaria. Dentro de la comunidad
universitaria encontramos: Estudiantes de Pregrado, Estudiantes de
Postgrado, Estudiantes de Cursos de Extensi\'on, Estudiantes del Ciclo
de Iniciaci\'on Universitaria, Estudiantes del Programa PIO,
Profesores y Empleados. Dentro de las dependencias que ofrecen
servicios en la universidad encontramos: Direcci\'on de Admisi\'on y
Control de Estudios (\emph{DACE}), Direcci\'on de Servicios,
Direcci\'on de Servicios Telem\'aticos, Biblioteca Central, entre
otros.
\newline
\newline
\indent Para los servicios nos interesar\'ia conocer: el nombre del
servicio, dependencia que lo ofrece, tipo de servicio, etc. Por
ejemplo: la Direcci\'on de Servicios ofrece el Servicio de
Comedores. Como es conocido, la USB es una instituci\'on de
Educaci\'on Superior P\'ublica, por lo tanto, es su deber ofrecer
alimentaci\'on a la comunidad que en ella habita. Para esto, la
Universidad mantiene hoy en d\'ia tres comedores principales: El
comedor ubicado al edificio de Matem\'aticas y Sistemas (\emph{MyS}),
el comedor de Casa del Estudiante, y el Comedor de Casa del
Empleado. Para los comedores nos interesar\'ia conocer: Nombre del
Comedor, Ubicaci\'on del Comedor, N\'umero de empleados del
comedor. Para las comidas que se sirven en los comedores se dispone de
un men\'u diario. Este men\'u lo realiza la Direcci\'on de
Servicios. Para el Men\'u nos interesa saber: Plato Principal, Bebida,
Postre, Comedor donde se sirve el plato. 
\newline
\newline
\indent Por otro lado, para que la comunidad acceda a este servicio,
es necesario que se disponga de un servicio de recarga de saldo para
pagar por el uso de los comedores. Para este servicio se podr\'ia
requerir los siguientes datos: Informaci\'on b\'asica de la persona
que requiere hacer la recarga del saldo, para esto es necesario
saber si la persona es estudiante, la informaci\'on a requerir es el
carnet, nombre y apellido, carrera que estudia. Si es Profesor, los
datos a requerir son: Carnet, Nombre y Apellido, Departamento
Asociado. Si es empleado, los datos a pedir son el carnet, nombre y
apellido, Dependencia donde trabaja. Adicionalmente, se va a requerir
de las personas la informaci\'on relacionada con la cuenta bancaria
donde se va a debitar el monto a recargar. Por \'ultimo, se va a
requerir el monto a recargar en el saldo respectivo.
\newline
\newline  
\indent Una posible requerimiento para la aplicaci\'on podr\'ia ser
poder ver el men\'u de la semana. Otro requerimiento podr\'ia ser,
cu\'al(es) es(son) el(los) comedor(es) m\'as usado(s) y los menos
usado(s), cu\'antos estudiantes  en total y en promedio usan el servicio de comedores por
d\'ia.
\newline
\newline
\indent Otro requerimiento que pudiera tener el servicio de comedores
es que una persona pueda en cualquier momento consultar su saldo,
as\'i como tamb\'en la recarga del mismo.
\newline
\newline
\indent Otra dependencia fundamental que existe dentro de la
Universidad es la Biblioteca Central. En este lugar se ofrecen
diversos servicios a la comunidad universitaria. Uno de esos servicios
es el de \emph{Pr\'estamo de Libros}. Como es conocido, la Biblioteca
Central contiene un gran n\'umero de Libros, Revistas Cient\'ificas,
Publicaciones Diversas que son consultadas por la comunidad, en
especial por los estudiantes para la realizaci\'on de sus diversas
tareas, o tambi\'en para el estudio de una asignatura
espec\'ifica. Para que esto sea posible, las personas pueden solicitar
el pr\'estamo de un ejemplar. Para ello se van a requerir los
siguientes datos: Los datos personales de la persona que solicita el
pr\'estamo,  T\'itulo del libro, Autor(es),
C\'odigo ISBN, Editorial, N\'umero de Edici\'on, Fecha de salida,
Fecha de Retorno. Existen distintos tipos de pr\'estamos, y \'estos se
determinan por la cantidad de d\'ias que dure el mismo. Esto se debe a
la ubicaci\'on del Libro. Si el libro se encuentra en el Departamento
de Pr\'estamos Circulante, el servicio dura un tiempo corto. En caso
contrario, dura un tiempo mayor. Es importante que todo libro debe ser devuelto a
m\'as tardar en la Fecha de Retorno. Si la persona no retorna el libro
prestado antes de la Fecha de Retorno, se le impondr\'a una multa que
ser\'a calculada de acuerdo a la cantidad de d\'ias que transcurra
desde la Fecha de Retorno hasta el momento en que la persona acuda a
devolver el Libro Prestado. Tambi\'en, como consecuencia el status de
la persona cambia, es decir, que no podr\'a solicitar pr\'estamos por
un per\'iodo determinado.   
\newline
\newline
\indent Una posible consulta para la aplicaci\'on podr\'ia ser que el
estudiante pueda verificar los ejemplares que est\'an disponibles y su
ubicaci\'on. Algo interesante con el requerimiento anterior ser\'ia
que junto a los ejemplares disponibles, la persona tenga la opci\'on
de seleccionar los que quiere reservar, y luego simplemente pasar por
la recepci\'on de la Biblioteca para la entrega de los libros
solicitados.
\newline
\newline
\indent Otro requerimiento podr\'ia ser consultar los libros
m\'as prestados y los menos prestados durante un per\'iodo
determinado. 
\newline
\newline
\indent Otro posible requerimiento podr\'ia ser que el estudiante
chequee en linea los libros que posee, as\'i como tambi\'en las Fechas
de Retorno de cada uno. Tambi\'en ser\'ia interesante que pudiera
chequear si posee multas por retraso, y tener la posibilidad de pagar
dicha multa de manera On-Line. Para esto se va a requerir toda la
informaci\'on concerniente a una cuenta bancaria de dicha persona de
donde se pueda debitar la deuda. Adicionalmente a esto, ser\'ia muy
util poder conocer su status, es decir, si esta autorizado para
solicitar pr\'estamos o si no puede hacer nuevas solicitudes por un
per\'iodo determinado.
\newline
\newline
El \'unico grafico que habia que subir era el logo de la universidad,
asi que voy a subir el archivo para que en la direccion que coloqu\'e
coloquen una local en su computadora.
INFO SOBRE LATEX: 
\newline
Lo unico relevante que tienen que saber es el uso de los acentos, que
como se habran dado cuenta se coloca el backslash '\' seguido de una
comilla simple y eso antes de la letra que le quieren poner acento. No
hace falta que tengan que hacer enter para pasar a la siguiente linea,
pueden escribir todo seguido, que latex hace la magia de ponerlo
bonito. El ultimo detalle es cuando finalizan un p\'arrafo, usaremos
\newline
\newline
\indent para que como se den cuenta hay dos lineas de por medio, y el
otro para la sangr\'ia. Ultimo detalle, para escribir la enne,
\~n. Es todo lo que deben saber para escribir. Si quieren poner
cursivas lo pueden hacer con \emph{Ejemplo cursivas}. Por ultimo, y
muy importante, para generar el pdf: pdflatex Fase1.tex (O el nombre
del archivo como lo tengan en la compu.)

\newpage
\section{Punto de Vista de Terceros}



\end{document}

