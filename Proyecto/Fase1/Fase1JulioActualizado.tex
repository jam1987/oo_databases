\documentclass[12pt,a4paper,spanish]{article}
\usepackage{babel}
\usepackage[latin1]{inputenc}
\usepackage[pdftex]{graphicx}

\begin{document}

\begin{figure}
  \centering
    \includegraphics{/Users/Julio/Downloads/logo.jpg}
     \\ Universidad Sim\'on Bol\'ivar
     \\ Depto. De Computaci\'on y Tecnolog\'ia de la Informaci\'on
     \\ CI5311: Paradigmas de Modelado de Bases de Datos
\end{figure} 

\title{Universo del Discurso}
\author{Grace Gim\'on 08-10437 \\
        Jes\'us Mart\'inez \\
        Julio De Abreu 05-38072}

\maketitle

\newpage

\indent Se quiere implementar una Base de Datos para la p\'agina Web
\emph{USBline}, a desarrollarse por la Universidad Sim\'on Bol\'ivar
para la gesti\'on y pago en l\'inea de servicios ofrecidos a toda la
comunidad universitaria y externa.
\newline
\newline
\indent La universidad y organizaciones terceras prestan numerosos servicios, pero la comunidad atendida es extensa
 dificultando una prestaci\'on eficiente en todo momento de estos. Por lo tanto la 
incorporaci\'on de un sistema web para la realizaci\'on de una solicitud o pago de un servicio 
agilizar\'ia y facilitar\'ia el d\'ia a d\'ia de todos aquellos que hacen vida 
en la universidad. Para ello, es imprescindible la creaci\'on de una base de datos, 
ya que s\'olo a trav\'es de ella se podr\'an almacenar integralmente los
datos de todos los servicios involucrados, permitiendo adem\'as realizar consultas relevantes
a los interesados.
\newline
\newline
\indent Un servicio es un conjunto de actividades que buscan
satisfacer las necesidades de una comunidad en particular. Un servicio
puede ser pago o gratuito. Dentro de la USB existen dos tipos de
servicios: Servicios ofrecidos por la misma universidad, y servicios
ofrecidos por terceros dentro de la misma.

\section{Punto de Vista de la Universidad}

\indent Dentro de la USB existen dependencias que ofrecen distintos
servicios a la comunidad universitaria. Dentro de la comunidad
universitaria encontramos: Estudiantes de Pregrado, Estudiantes de
Postgrado, Estudiantes de Cursos de Extensi\'on, Estudiantes del Ciclo
de Iniciaci\'on Universitaria, Estudiantes del Programa PIO,
Profesores y Empleados. Dentro de las dependencias que ofrecen
servicios en la universidad encontramos: Direcci\'on de Admisi\'on y
Control de Estudios (\emph{DACE}), Direcci\'on de Servicios,
Direcci\'on de Servicios Telem\'aticos, Biblioteca Central, entre
otros.
\newline
\newline
\indent Para los servicios nos interesar\'ia conocer: el nombre del
servicio, dependencia que lo ofrece, tipo de servicio, etc. Por
ejemplo: la Direcci\'on de Servicios ofrece el Servicio de
Comedores. Como es conocido, la USB es una instituci\'on de
Educaci\'on Superior P\'ublica, por lo tanto, es su deber ofrecer
alimentaci\'on a la comunidad que en ella habita. Para esto, la
Universidad mantiene hoy en d\'ia tres comedores principales: El
comedor ubicado al edificio de Matem\'aticas y Sistemas (\emph{MyS}),
el comedor de Casa del Estudiante, y el Comedor de Casa del
Empleado. Para los comedores nos interesar\'ia conocer: Nombre del
Comedor, Ubicaci\'on del Comedor, N\'umero de empleados del
comedor. Para las comidas que se sirven en los comedores se dispone de
un men\'u diario. Este men\'u lo realiza la Direcci\'on de
Servicios. Para el Men\'u nos interesa saber: Plato Principal, Bebida,
Postre, Comedor donde se sirve el plato. 
\newline
\newline
\indent Por otro lado, para que la comunidad acceda a este servicio,
es necesario que se disponga de un servicio de recarga de saldo para
pagar por el uso de los comedores. Para este servicio se podr\'ia
requerir los siguientes datos: Informaci\'on b\'asica de la persona
que requiere hacer la recarga del saldo, para esto es necesario
saber si la persona es estudiante, la informaci\'on a requerir es el
carnet, nombre y apellido, carrera que estudia. Si es Profesor, los
datos a requerir son: Carnet, Nombre y Apellido, Departamento
Asociado. Si es empleado, los datos a pedir son el carnet, nombre y
apellido, Dependencia donde trabaja. Adicionalmente, se va a requerir
de las personas la informaci\'on relacionada con la cuenta bancaria
donde se va a debitar el monto a recargar. Por \'ultimo, se va a
requerir el monto a recargar en el saldo respectivo.
\newline
\newline  
\indent Una posible requerimiento para la aplicaci\'on podr\'ia ser
poder ver el men\'u de la semana. Otro requerimiento podr\'ia ser,
cu\'al(es) es(son) el(los) comedor(es) m\'as usado(s) y los menos
usado(s), cu\'antos estudiantes  en total y en promedio usan el servicio de comedores por
d\'ia.
\newline
\newline
\indent Otro requerimiento que pudiera tener el servicio de comedores
es que una persona pueda en cualquier momento consultar su saldo,
as\'i como tamb\'en la recarga del mismo.
\newline
\newline
\indent Otra dependencia fundamental que existe dentro de la
Universidad es la Biblioteca Central. En este lugar se ofrecen
diversos servicios a la comunidad universitaria. Uno de esos servicios
es el de \emph{Pr\'estamo de Libros}. Como es conocido, la Biblioteca
Central contiene un gran n\'umero de Libros, Revistas Cient\'ificas,
Publicaciones Diversas que son consultadas por la comunidad, en
especial por los estudiantes para la realizaci\'on de sus diversas
tareas, o tambi\'en para el estudio de una asignatura
espec\'ifica. Para que esto sea posible, las personas pueden solicitar
el pr\'estamo de un ejemplar. Para ello se van a requerir los
siguientes datos: Los datos personales de la persona que solicita el
pr\'estamo,  T\'itulo del libro, Autor(es),
C\'odigo ISBN, Editorial, N\'umero de Edici\'on, Fecha de salida,
Fecha de Retorno. Existen distintos tipos de pr\'estamos, y \'estos se
determinan por la cantidad de d\'ias que dure el mismo. Esto se debe a
la ubicaci\'on del Libro. Si el libro se encuentra en el Departamento
de Pr\'estamos Circulante, el servicio dura un tiempo corto. En caso
contrario, dura un tiempo mayor. Es importante que todo libro debe ser devuelto a
m\'as tardar en la Fecha de Retorno. Si la persona no retorna el libro
prestado antes de la Fecha de Retorno, se le impondr\'a una multa que
ser\'a calculada de acuerdo a la cantidad de d\'ias que transcurra
desde la Fecha de Retorno hasta el momento en que la persona acuda a
devolver el Libro Prestado. Tambi\'en, como consecuencia el status de
la persona cambia, es decir, que no podr\'a solicitar pr\'estamos por
un per\'iodo determinado.   
\newline
\newline
\indent Una posible consulta para la aplicaci\'on podr\'ia ser que el
estudiante pueda verificar los ejemplares que est\'an disponibles y su
ubicaci\'on. Algo interesante con el requerimiento anterior ser\'ia
que junto a los ejemplares disponibles, la persona tenga la opci\'on
de seleccionar los que quiere reservar, y luego simplemente pasar por
la recepci\'on de la Biblioteca para la entrega de los libros
solicitados.
\newline
\newline
\indent Otro requerimiento podr\'ia ser consultar los libros
m\'as prestados y los menos prestados durante un per\'iodo
determinado. 
\newline
\newline
\indent Otro posible requerimiento podr\'ia ser que el estudiante
chequee en linea los libros que posee, as\'i como tambi\'en las Fechas
de Retorno de cada uno. Tambi\'en ser\'ia interesante que pudiera
chequear si posee multas por retraso, y tener la posibilidad de pagar
dicha multa de manera On-Line. Para esto se va a requerir toda la
informaci\'on concerniente a una cuenta bancaria de dicha persona de
donde se pueda debitar la deuda. Adicionalmente a esto, ser\'ia muy
util poder conocer su status, es decir, si esta autorizado para
solicitar pr\'estamos o si no puede hacer nuevas solicitudes por un
per\'iodo determinado.
\newline
\newline
\indent Otro servicio que se relaciona mucho con los procesos que
realiza DACE, es el de los cursos ofrecidos por el Decanato de
Extensi�n.Como es conocido, existen otras dependencias que hacen vida
dentro de la USB, como lo son los Decanatos. Entre ellos podemos
mencionar: El Decanato de Estudios Generales, El Decanato de
Investigaci\'on, y Desarrollo, El Decanato de Extensi\'on, entre
otros. El Decanato de Extensi\'on se encarga de todos los procesos
externos de la USB, como lo son el Programa de Pasant\'ias, Servicio
Comunitario, Cursos de Idiomas, Programas de Emprendimiento, etc. Para
que los cursos de Idiomas que son ofrecidos por este decanato se
puedan llevar a cabo, es necesario el apoyo que es brindado por la
Direci\'on de Admisi\'on y Control de Estudios (\emph{DACE}), ya que
es esta dependencia la que maneja la disponibilidad de cupos, la
disponibilidad de aulas para la realizaci\'on de los
mismos. Actualmente la inscripci\'on de los cursos se realiza
manualmente y adem\'as el pago del arancel es obligatoriamente un
dep�sito, por lo tanto, la inclusi\'on en la base de datos planteada,
de toda la informaci\'on referente al servicio de cursos de
extensi\'on, facilitar\'ia a toda la comunidad USB a la hora de
inscribir un curso.  Estos cursos pueden ser de Idiomas (Ingles, Italiano, Chino,
 Japon\'es, Franc\'es, Ruso, \'Arabe, Portugu\'es) y cursos de otra
 \'indole. Por cada curso necesitaremos el c�digo de la materia,
 nombre, descripci�n, contenido, monto y la cantidad de cupos
 posibles. Cada curso posee asignado un profesor, un aula y
 horario. De los profesores, nos interesa su c\'edula y nombre
 completo, del horario, los d\'ias y horas en que se imparte el
 curso. Las aulas ya est\'an contempladas por el proceso de
 inscripci\'on de los estudiantes en DACE. Aquellas que al finalizar
 dicho proceso permanezcan disponibles, ser\'an entonces asignadas a
 este ente. Por otra parte, los cursos pertenecen a un per\'iodo del
 a�o, entonces nos interesar\'ia saber en qu� trimestres suelen
 ofertarse ciertas asignaturas. 
\newline
\newline
\indent Los cursos del Decanato de Extensi\'on no est�n reservados
\'unicamente a los estudiantes de la universidad. Esto significa que
cualquier persona puede inscribirse en un curso ofrecido. Por lo
tanto, la informaci\'on que nos interesa de los inscritos, son su
c\'edula, nombre completo, fecha de nacimiento, tel\'efonos de
contacto (celuaar y casa) y un correo electr\'onico de uso
frecuente. Para la inscripci\'on de los cursos se deben cumplir
requisitos, por ejemplo, si se desea inscribir Ingl�s II, se debe
tener el conocimiento impartido en Ingl�s I. Nos interesa entonces
tener la informaci\'on de los requisitos que posee un curso, los
requisitos pueden ser otros cursos anteriores o el resultado de una
prueba de nivelaci\'on, de �sta \'ultima nos interesa saber la fecha
de su realizaci\'on y calificaci'on obtenida. Tambi\'en ser�a �til
conocer los materiales o libros que se van a emplear en el curso,
d\'onde se pueden conseguir y su costo estimado en tiempo real. Otro
elemento interesante ser\'ia mantener un registro de los cursos
realizados por las personas, para as\'i facilitar el proceso de
inscripci\'on, ya que de esa manera, se puede obtener los cursos
aprobados para el momento que inscriba un nuevo curso que tenga como
requisito otra asignatura antes cursada. 
\newline
\newline
\indent Durante el per\'iodo de inscripci\'on entonces se debe poder
consultar la informaci\'on m�s relevante sobre un curso que est\'e
abierto. En particular, nos interesa saber si quedan cupos libres en
tiempo real, en el caso de no haber, el curso debe ser marcado como lleno y no permitir el ingreso de nuevos inscritos. 
\newline
\newline
\indent Este m\'odulo del sistema, debe tener la capacidad de
responder las inquietudes m\'as b\'asicas a la hora de interesarse por
un curso, entre esas podemos decir: B\'usqueda de un curso por
categor�a, nombre, profesor, c\'odigo, cantidad de cupos libres
actualmente. La inscripci\'on debe realizarse online en alg\'un curso
que no est\'e lleno, el pago del curso podr\'ia realizarse por
transferencia a la cuenta Mercantil o cualquier otro mecanismo de pago
electr\'onico ofrecido por el banco.
\newline
\newline
\indent Para las aulas donde ser\'an impartidos estos cursos, es necesario conocer: capacidad, disponibilidad, ubicacion y N\'umero.



El \'unico grafico que habia que subir era el logo de la universidad,
asi que voy a subir el archivo para que en la direccion que coloqu\'e
coloquen una local en su computadora.
INFO SOBRE LATEX: 
\newline
Lo unico relevante que tienen que saber es el uso de los acentos, que
como se habran dado cuenta se coloca el backslash '\' seguido de una
comilla simple y eso antes de la letra que le quieren poner acento. No
hace falta que tengan que hacer enter para pasar a la siguiente linea,
pueden escribir todo seguido, que latex hace la magia de ponerlo
bonito. El ultimo detalle es cuando finalizan un p\'arrafo, usaremos
\newline
\newline
\indent para que como se den cuenta hay dos lineas de por medio, y el
otro para la sangr\'ia. Ultimo detalle, para escribir la enne,
\~n. Es todo lo que deben saber para escribir. Si quieren poner
cursivas lo pueden hacer con \emph{Ejemplo cursivas}. Por ultimo, y
muy importante, para generar el pdf: pdflatex Fase1.tex (O el nombre
del archivo como lo tengan en la compu.)

\newpage
\section{Punto de Vista de Terceros}

\indent En la universidad, las empresas pueden tener un espacio para
ofrecer sus servicios. Entre los cuales nos ser\'an de inter\'es el
Banco Mercantil y Xerox: Centro de impresiones y fotocopias. Estos pueden ser utilizados tanto por la comunidad de la USB como cualquier persona visitante a la universidad. 
\newline
\newline
\indent Xerox es un establecimiento en el cual se pueden realizar
impresiones de diversos formatos, fotocopias, se ofrecen algunos
art\'iculos de papeler\'ia, electr\'onicos, CDs y adem\'as es un
centro computarizado, que cuenta con unas 15 computadoras por las
cuales es posible acceder a internet. Por otra parte, ofrece el
servicio de reproducci\'on de gu\'ias, extractos de libros o ensayos
que los profesores utilizan en sus clases, para todos los estudiantes
que lo deseen, sin embargo tambi\'en los estudiantes pueden contribuir
esta lista de materiales acad\'emicos.
\newline
\newline
\indent El centro de fotocopiado e impresiones es utilizado por muchas
personas al d\'ia. Los estudiantes imprimen trabajos, gu\'ias,
fotocopian cuadernos, acceden a internet. Los profesores tambi\'en
pueden hacer este tipo de actividades y adem\'as pueden solicitar
impresi\'on de ex\'amenes, gu'ias, materiales por encargo. Los
visitantes a la universidad pueden acceder de la misma manera a
cualquiera de los servicios ofertados.
\newline
\newline
\indent Los estudiantes y profesores son los \'unicos permitidos en
encargar impresiones o fotocopias espec\'ificas. De ambos se debe
registrar un correo electr\'onico, de los estudiantes nos interesa
saber el carnet y del profesor su carnet y el Departamento Asociado.
\newline
\newline
\indent Los servicios ofertados en Xerox son: Fotocopias e
impresiones, venta de art\'iculos como: engrapadoras, sobres, hojas,
las gu\'ias y materiales dejados para impresi\'on y alquiler de
computadoras. De los art\'iculos de papeler\'ia, nos interesan su
nombre, descripci\'on, costo y cantidad en stock.
\newline
\newline
\indent Las impresiones pueden ser Blanco/Negro, Color, hoja oficio,
carta y otros, nos interesa su descripci\'on y costo. Las fotocopias
poseen un monto fijo por hoja. Las personas al hacer uso de este
servicio piden un n\'umero y esperan su turno, por lo tanto lo que nos
interesa es la cantidad de personas que est�n en espera. 
\newline
\newline
\indent Con respecto a las gu\'ias dejadas para reproducci\'on, nos
interesa nombre, descripci\'on, c\'odigo de la materia a la que est\'a
relacionada, nombre de la persona que lo entreg\'o y el costo de la
reproducci\'on. El proceso de impresiones y fotocopias puede tener dos
modalidades: la persona puede solicitar la impresi\'on de un trabajo
el cual puede cargar al sistema  indicando el n\'umero de copias que
desea, y el otro modo es seleccionando de las gu\'ias y trabajos que
ya est\'an en impresi\'on, alguno e indicar n\'umero de copias que
desea. El sistema calcula, dependiendo de la cantidad de personas que
est\'an en espera por impresiones y fotocopias, calcula un tiempo
estimado para que el interesado vaya a recoger las fotocopias.
\newline
\newline
\indent Con respecto a la cancelaci\'on de las fotocopias e
impresiones, los estudiantes tienen dos opciones para hacerlo: la
primera es solicitando una cantidad fija de fotocopias pag\'andolas
por prepago por medio del Banco Mercantil y siendo acreditadas a la
cuenta de la TAI (Tarjeta Acad�mica Inteligente), la segunda opci\'on
es en l\'inea por medio de la tarjeta prepago del Banco Mercantil o en
efectivo al recoger las fotocopias.
\newline
\newline
\indent Para el alquiler de computadoras, necesitamos saber el
n\'umero de computadoras disponibles para uso, computadoras en
funcionamiento, costo del alquiler por hora y por fracci\'on de hora. 
\newline
\newline
\indent Considerando que se posee almacenada toda esta informaci\'on, las inquietudes m\'as b\'asicas que se desea puedan ser respondidas
son: Si el centro est\'a abierto o cerrado, si hay computadoras
disponibles para su uso, cu\'antas personas hay en espera por el
servicio de fotocopiado e impresiones. Adem�s de la b\'usqueda de
cualquier servicio por cualquiera de sus propiedades.
\newline
\newline
\indent Una consulta muy \'util podr\'ia ser que las personas pudieran
saber que gu\'ias se encuentran disponibles en el stock. Para eso se
podr\'ia aplicar una b\'usqueda por categor\'ia, por c\'odigo de
asignaturas, por nombre de profesores, por temas, etc.


\end{document}

